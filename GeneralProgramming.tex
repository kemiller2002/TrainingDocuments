\documentclass {amsart}

\usepackage{graphicx}
\usepackage{listings}
\usepackage{color}
\usepackage{verbatim}
\usepackage{hyperref}



\title{General Programming Principles}


\begin{document}
\maketitle

\includegraphics[scale=6]{GOTO.png}


\section{SOLID Principles}
	\subsection{links}
		\begin{itemize}
			\item Wikepedia \url{http://en.wikipedia.org/wiki/Solid_(object-oriented_design)}
		\end{itemize}

\section{Functional Programming}
	\subsection{Void} Avoid it.
		Sometimes methods call for using void.  It's a function that performs a side effect and doesn't explicitly say that it was successful.	
	\subsection{Parameters}
		\begin{itemize}
			\item Ideally, no more than three.  Absolutely, no more than five.
			\item ref and out.  Avoid using these.  It's better to return a value than modify an existing value.
			\item Avoid modifying parameters that are passed into a function.  The calling procedure has no idea that it's been changed. 
		\end{itemize}
\section{Coding Stlyles}
	\subsection {Conditionals} when evaulating and returning conditionals

		\begin{verbatim}
		if(boolean)
		\end{verbatim}

		and not 
		
		\begin{verbatim}
		if(boolean == true)
		\end{verbatim} 
		
		or 
		
		\begin{verbatim}
		if(boolean != false)
		\end{verbatim}

		also do not do
		\begin{verbatim}
		if(boolean)
		{
		     return true;
		}
		else
		{
		     return false;
		}
		\end{verbatim}
		or
		\begin{verbatim}
		return (boolean == true);
		\end{verbatim}

		
\section{Exceptions}
	\subsection{Stack Trace}  Always get the stack trace.

	\includegraphics[scale=.45]{StackTrace.jpg}
	\subsection{Catching}  Never catch an exception and do nothing with it.  




\end{document}